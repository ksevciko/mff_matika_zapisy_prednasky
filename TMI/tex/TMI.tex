\documentclass{article}
\usepackage[utf8]{inputenc}		% For UTF-8 encoding
\usepackage[IL2]{fontenc}		% Czech font encoding
\usepackage[czech]{babel}		% Czech language support
\usepackage{amsmath}			% For advanced math typesetting
\usepackage{amsfonts}			% For additional fonts
\usepackage{amssymb}			% For additional symbols
\usepackage{amsthm}				% For theorem environments
\usepackage{geometry}			% To adjust page margins
\usepackage{hyperref}			% For clickable links and references
\usepackage[dvipsnames]{xcolor}	% For colors - highliting
\usepackage{soul}				% For command hl = highlight
\usepackage{enumerate}			% For lists
\usepackage{soulpos}

\DeclareRobustCommand{\hldef}[1]{{\sethlcolor{SkyBlue}\hl{#1}}}
\DeclareRobustCommand{\hldefmath}[1]{{\colorbox{SkyBlue}{#1}}}
\DeclareRobustCommand{\hlveta}[1]{{\sethlcolor{magenta}\hl{#1}}}
\DeclareRobustCommand{\hlvetamath}[1]{{\colorbox{magenta}{#1}}}

% Set page margins
\geometry{a4paper, margin=1in}

% Define theorem environment
\newtheorem{veta}{Věta}[section]
% \newtheorem{vetal}[vetat]{Věta L}
\newtheorem*{tvrzeni}{Tvrzení}
\newtheorem*{lemma}{Lemma}
\newtheorem*{dusledek}{Důsledek}
\newtheorem*{pozn}{Poznámka}
\newtheorem*{pozorovani}{Pozorování}
\newtheorem*{pr}{Příklad}
\theoremstyle{definition}
\newtheorem*{definice}{Definice}

\title{Matematická analýza}
\author{Kateřina Ševčíková, podle učebního textu profesora Jana Rataje}
\date{Poslední úprava: \today}

\begin{document}

\maketitle

\tableofcontents
\newpage

\section*{Úvod}
Připomenutí: Riemannův, Newtonův integrál, geometrický význam plochy pod
grafem. (Riemannův integrál lze použít k výpočtu míry = integrálu, ale jen na 
uzavřeném intervalu a pro omezenou funkci.)
(( 

Ne všechny funkce jsou ”integrovatelné”, ne všechny množiny ”měřitelné”.
\textit{úplnost} :  Na  prostoru  Riemannovsky  integrovatelných  funkcí  na  intervalu  \(I\) 
definujme skalární součin vztahem   \( \langle f, g\rangle    :=   \int _I  f \cdot g \). 
Indukovaný metrický prostor není úplný.

\textit{aditivita}: V teorii pravděpodobnosti potřebujeme, aby pravděpodobnostní míra  
byla  spočetně aditivní,  tedy  aby  pro  po  dvou  disjunktní  náhodné jevy \(A_1, A_2, . . .\) 
platilo \(Pr(\bigcup _i A_i) = \sum_i Pr(A_i)\).
Toto by pro míru definovanou pomocí Riemannova integrálu neplatilo.

Obecná konstrukce: nejprve  míra (množinová  funkce), z ní  je  odvozen  integrál (aproximace po částech konstantními funkcemi).
Vlastnosti, které chceme po ”míře”:
\begin{enumerate}
    \item \( \mu (\emptyset) = 0, \mu(A) \geq 0 \ \forall A \)
    \item \(\mu(\bigcup _n A_n) = \sum _n \mu(A_n)\) pro po dvou disjunktní množiny \(A_1, A_2, ... \)
\end{enumerate}

 ))

Banachův - Tarského paradox: u míry chceme moct rozdělit množinu na několik částí, 
každou posunout, zrotovat, tak ať jou stále disjunktní, a chceme mít staále stejnou míru.
To ale vždy nefunguje, tento paradox ukazuje, že je možné rozdělit jednotkovou kouli v \(\mathbb{R}^3\) 
na 5 částí, posunout je, a získat 2 stejné koule, tedy nezachováme míru.
Tedy ne každá množina je měřitelná.

\section{Základní pojmy teorie míry}
\begin{veta}
    \hlvetamath{Existence nejmenší \(\sigma\)-algebry}
\end{veta}
\begin{veta}
    slkdjl
\end{veta}
\begin{definice}
    Nechť \(a = (a_1, ..., a_n), b = (b_1, ..., b_n) \in \mathbb{R}\). Množina \(W = {x = (x_1, ..., x_n) \in \mathbb{R}:
    a_i < x_i < b<i pro všechna i \in {1, ..., n}}\) a také ksždou množinu, která vznikne záměnou libovolného znaménka "<" za "<=", 
    nazveme \hldef{n-buňka}. \hldef{Objem} n-buňky definujeme jako 0, je-li \(W = \emptyset\) a jako \( vol(W) = \prod_ ...\)
\end{definice}
\begin{veta}
    \hlveta{Rozšíření elementárního objemu}
Existuje
\end{veta}
\begin{proof}
    Náznak: Lze ukázat, že je-li \(G \in \mathbb{R}\) otevřená, pak existují po dvou disjunktní n-buňky takové, že 
    \(G = \bigcup_{i = 1}^{\infty} W_i\).
    Definujeme \(Z_n (G) = \sum_{i=1}^{\infty}vol(W_i).\)
    (nezáleží na volbě rozkladu). Dále pak \(A \in \mathcal{B} (\mathbb{R}^n)\) definujeme
    \(Z_n (A) = inf\)\{\(Z_n (G): G\) otevřená, \(G \in \mathbb{R}^n, A \subset G\)\}. 
\end{proof}
\begin{pozn}
\begin{itemize}
    \item Z konstrukce míry \(Z_n\) plyne, že je-li \(A \subset \mathbb{R}^n\) borelovská a 
    \(\epsilon < 0\), potom existuje otrevřená množina \(G \in \mathbb{R}^n taková, že A \subset G a Z_n(G \setminus A) < A\) 
    \item Míra \(Z_n\) je invariantní vůči posunutí - pro všechna \( x \in \mathbb{R}^n\) a \(A \in \mathcal{B}(\mathbb{R}^n)\) 
    platí !!!!!!!!!!
\end{itemize}
\end{pozn}
\begin{definice}
Nechť \((X, \mathcal{A} , \mu)\) je prostor s mírou. Řekneme, že \(\mu\) je \hldef{úplná míra}, jestliže platí:
je-li \(A \in \mathcal{A} \) splňující \(\mu (A) = 0\) a \(A^I \in A\) , pak \(A^I \in \mathcal{A} \).
\end{definice}
\begin{pozn}
    V takovém případě nutně \(\mu (A^I) - 0\) !!!!!!!!!!!!!!!!!
\end{pozn}
\begin{veta}
\hlveta{Zúplnění míry} (bez dk)

Nechť \((X, \mathcal{A} , \mu)\) je prostor s mírou. Nechť \(\mathcal{A}_0\) je systém všech množin 
\(E subset X\), pro něž existují \(A, B \in \mathcal{A} \) takové, že \(A \subset E \subset B a \mu(B \setminus A) = 0\). Potom 
\(\mathcal{A}_0\) je \(\sigma\)-algebra obsahující \(\mathcal{A}\). 
Definujeme \(\mu_0(E) = \mu(A) \forall E \in \mathcal{A}_0\). Potom \(\mu = \mu_0\) 
na \(\mathcal{A} \) a \((X, \mathcal{A}_0 , \mu_0)\) je prostor s úplnou mírou.
\end{veta}
\begin{definice}
Prostor  !!!!!!!!!!!!!!!!!!!!!!!!!!!!!!
\end{definice}
\begin{definice}
    Zúplnění \(\sigma\)-algebry \(\mathcal{B} (\mathbb{R}^n)\) vzhledem i \(Z_n\) značíme \(\mathcal{B}_0(\mathbb{R}^n) \)
    a nazýváme ji \(\sigma\)-algebrou \hldef{lebesgueovsky měřitelných množin}. Odpovídající zúplnění míry \(Z_n\) značíme opět \(Z_n\) 
    a nazýváme je hldef{Lebesgueovou mírou}.
\end{definice}

\section{Měřitelné funkce}
\begin{definice}
    Nechť \((X, \mathcal{A})\) je měřitelný prostor a \((Y, \tau)\) je metrický prostor. 
    Řekneme, že zobrazení \(f: X -> Y\) je \hldef{měřitelné}, jestliže \(f^{-1} \in \mathcal{A}\) 
    pro každou \(V \subset Y\) otevřenou. Je-li navíc \((X, \rho)\)metrický prostor a \(\mathcal{A} = \mathcal{B}(X)\) 
    , pak F nazýváme hldef{borelovské}.
\end{definice}
\begin{pozn}
    Nechť \((X, \rho), (Y, \tau)\) jsou M.P.. Pak zobrazení \(g: X -> Y\) je spojité práve tehdy když 
    \(g^{-1}(V)\) je otevřená v X pro každou V otevřenou v Y.
    Tedy každé spojité zobrazení je borelovké.
\end{pozn}
\begin{pr}
Nechť \((X, \mathcal{A})\) je měřitelný prostor, \(A \subset X\). Potom \hldef{charakteristická funkce} 
množiny A je definovaná předpisem \(\mathcal{x}_A (x) = 1, pokud x \in A, 0, pokud x \notin A\) je 
měřitelné práve tehdy, když \(A \in \mathcal{A}\)
\end{pr}
\begin{proof}
    ,,=>" Je-li \(\mathcal{x}_A\) měřitelná, pak \(A = \mathcal{x}_A^{-1} ((1/2, 3/2))\) je vzor otevřené množiny, a 
    tehdy \(A \in \mathcal{A}\)
    ,,<=" Nechť \(A \in \mathcal{A}, A \subset \mathbb{R}\) otevřená. Pak 
    \(\mathcal{x}_A^{-1} = \)
    \begin{itemize}
        \item X, pokud 0, 1 \(\in B\),
        \item A, pokud \(0 \notin B, 1 \in B\),
        \item \(X \setminus A\)
        !!!!!!!!!!!!!!!!!!!!!!
    \end{itemize}
    Dle vlastností \(\sigma\)-algebry patří všechny tyto množiny so \(mathcal{A}\), a tedy \(\mathcal{x}_A\) je měřitelná.
\end{proof}
\begin{veta}
    \hlveta{měřitelnost složení zobrazení}
    Nechť \((Y, \tau), (Z, \sigma) \) jsou M.P. a \((X, \mathcal{A})\) je měřitelný prostor. 
    Nechť \(g: Y -> Z\) je spojité a \( f:X->Y \) je měřitelné. Potom \(gof: X -> Z\) je měřitelné.
\end{veta}
\begin{proof}
    obrázkem
\end{proof}

\section{Funkce jedné reálné proměnné – derivace a Taylorův polynom}
Derivace a Taylor.

\section{Řady}
Konvergence řad.

\section{Primitivní funkce}
Primitivní funkce, integrace.

\section{Určitý integrál}
Riemannův a Newtonův integrál.

% %section 8: Obyčejné diferenciální rovnice, předmět matematická analýza, přednášky prof. Stanislav Hencl

\section{Obyčejné diferenciální rovnice}
= diferenciální rovnice s jednou proměnnou, s více proměnnými jsou to parciální diferenciální rovnice

\subsection{Řešení, existence a jednoznačnost}
\(y'(x) = f(x, y(x))\) má řešení, je-li f hezká

\begin{definice}
	Nechť \(\Phi : \Omega \subset \mathbb{R}^{n+2} \to \mathbb{R}\).\ \hldef{Obyčejnou diferenciální rovnicí} (zkratka ODR) 
	n-tého řádu nazveme 
\begin{equation}
	\Phi(x, y(x), y'(x), \ldots, y^{(n)}(x)) = 0
\end{equation}
\end{definice}

\begin{definice}
	\hldef{Řešení} obyčejné diferenciální rovnice ma otevřeném intervalu $I \subset \mathbb{R}$ je funkce splňující
\begin{enumerate}[(i)]
	\item existuje $y^{(k)}(x)$ vlastní pro $k = 1, 2, \ldots, n$ pro všechna $x \in I$ 
	\item rovnice (1) platí pro všechna $x \in I$
\end{enumerate}
	Řešením je dvojice $(y, I)$.
\end{definice}

\begin{definice}
	Řekneme, že $(\tilde{y}, \tilde{I})$ je \hldef{rozšířením} $(y, I)$, pokud
\begin{enumerate}[(i)]
	\item $\tilde{y}$ je řešení (1) na $\tilde{I}$
	\item $I \subset \tilde{I}$
	\item $y = \tilde{y}$ na $I$
\end{enumerate}
	Řekneme, že $(y, I)$, je \hldef{maximální řešení}, pokud nemá rozšíření.
\end{definice}

\begin{definice}
	Řekneme, že $I \subset \mathbb{R}^{n}$ je \hldef{otevřený interval}, pokud existují otevřené
	intervaly $I_1, I_2, \ldots, I_n$ tak, že $I = I_1 \times \cdots \times I_n$.
\end{definice}

\begin{definice}
	Nechť $c \in \mathbb{R}^{n}$ a $r>0$. Definujeme \hldef{otevřenou kouli} jako 
	\[ B(c, r) = \left\{ x \in \mathbb{R}^{n} : |x - c|  
	= \sqrt{\sum_{i = 1}^{n} (x_i - c_i)^{2}} > r \right\} \]
\end{definice}

\begin{definice}
	Nechť $I \subset \mathbb{R}^{n}$ je otevřený interval a $f: I \to \mathbb{R}$ je funkce. Řekneme, že $f$ je
	\hldef{spojitá} v bodě $x_0 \in I$, pokud $\forall \epsilon > 0 \  \exists \delta > 0 \ 
	\forall x \in B(x_0, \delta) \cap I$ platí $|f(x) - f(x_0)| < \epsilon$.
	Řekneme, že $f$ je spojitá na $I$, pokud je spojitá ve všech bodech I.
\end{definice}

\begin{pozorovani}
	...
\end{pozorovani}

\begin{proof}
	Důkaz pozorování.
\end{proof}

\begin{dusledek}
	\(P(x,y)\) polynom dvou proměnných je spojitá funkce na \(\mathbb{R}^{2}\)
\end{dusledek}

\begin{vetat}
	\hlvetamath{Peano s $y^{(n)}$} (důkaz později, ne tento semestr)
	\\
	Nechť \(I \subset \mathbb{R}^{n+1}\) je otevřený interrval, \(f: I \to \mathbb{R}\) je spojitá, a \linebreak[1]
	\([x_0, y_0, \dots, y_{n-1}] \in I\). Pak \linebreak[1] \(\exists \delta > 0\) a v okolí \linebreak[1]
	\(x_0\) existuje interval \linebreak[1] \((x_0- \delta , x_0 + \delta)\) a funkce \(y(x)\) definována na 
	\linebreak[1] \((x_0 - \delta, x_0 + \delta)\) tak, že \(y(x)\) splňuje ODR \linebreak[1]
	\(y^{(n)}(x) = f(x, y(x^), \dots, y^{(n-1)}) \forall x \in (x_0 - \delta, x_0 + \delta)\) s počáteční podmínkou 
	\(y(x_0) = y_0, y'(x_0) = y_1, \dots , y^{(n-1)}(x_0) = y_{n-1}\).
\end{vetat}

\begin{pozn}
\begin{enumerate}
	\item tato věta je lokální a \(\delta\) může být velmi malé
	\item tato věta nedává jednoznačnost řešení
	\item každé řešení lze rozšířit do maximálního řešení
\end{enumerate}
\end{pozn}

\begin{definice}
	\(f: \mathbb{R} \to \mathbb{R}\) je \hldef{lipschtzovská}, 
	pokud \(\exists K > 0\), že \(|f(x) - f(y)| \leq K |x - y| \ \  \forall x,y \in \mathbb{R}\).
\end{definice}

\begin{definice}
	Nechť \(I \subset \mathbb{R}^{2}\) je otevřený interval. Řekneme, 
	že \(f: I \to \mathbb{R}\) je \hldef{lokálně lipschitzovský vůči y}, 
	pokud \(\forall U \subset I\) omezenou \(\exists K \in \mathbb{R}\), že
	\(|f(x,y) - f(x, \tilde{y})| \leq K |y - \tilde{y}| \ \ \forall [x, \tilde{y}] \in U\).
\end{definice}

\begin{vetat}
	\hlveta{Picard}
	\\
	Nechť \(I \subset \mathbb{R}^{2}\) je otevřený interval a \([x_0, y_0] \in I\). Nechť \(f: I \to \mathbb{R}\)
	je spojitá a lokálně lipschitzovská vůči y. Pak existuje \((x_0 - \delta, x_0 + \delta)\) a funkce
	\(y(x): (x_0 - \delta, x_0 + \delta) \to \mathbb{R}\) tak, že \(y(x)\) splňuje ODR
	\(y'(x) = f(x, y(x))\) pro \(x \in (x_0 - \delta, x_0 + \delta)\) s počáteční podmínkou \(y(x_0) = y_0\).
	\\
	Navíc y je jedinné řešení na \((x_0 - \delta, x_0 + \delta)\).
\end{vetat}

\end{document}

