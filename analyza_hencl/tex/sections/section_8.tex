%section 8: Obyčejné diferenciální rovnice, předmět matematická analýza, přednášky prof. Stanislav Hencl

\section{Obyčejné diferenciální rovnice}
= diferenciální rovnice s jednou proměnnou, s více proměnnými jsou to parciální diferenciální rovnice

\subsection{Řešení, existence a jednoznačnost}
\(y'(x) = f(x, y(x))\) má řešení, je-li f hezká

\begin{definice}
	Nechť \(\Phi : \Omega \subset \mathbb{R}^{n+2} \to \mathbb{R}\).\ \hldef{Obyčejnou diferenciální rovnicí} (zkratka ODR) 
	n-tého řádu nazveme 
\begin{equation}
	\Phi(x, y(x), y'(x), \ldots, y^{(n)}(x)) = 0
\end{equation}
\end{definice}

\begin{definice}
	\hldef{Řešení} obyčejné diferenciální rovnice ma otevřeném intervalu $I \subset \mathbb{R}$ je funkce splňující
\begin{enumerate}[(i)]
	\item existuje $y^{(k)}(x)$ vlastní pro $k = 1, 2, \ldots, n$ pro všechna $x \in I$ 
	\item rovnice (1) platí pro všechna $x \in I$
\end{enumerate}
	Řešením je dvojice $(y, I)$.
\end{definice}

\begin{definice}
	Řekneme, že $(\tilde{y}, \tilde{I})$ je \hldef{rozšířením} $(y, I)$, pokud
\begin{enumerate}[(i)]
	\item $\tilde{y}$ je řešení (1) na $\tilde{I}$
	\item $I \subset \tilde{I}$
	\item $y = \tilde{y}$ na $I$
\end{enumerate}
	Řekneme, že $(y, I)$, je \hldef{maximální řešení}, pokud nemá rozšíření.
\end{definice}

\begin{definice}
	Řekneme, že $I \subset \mathbb{R}^{n}$ je \hldef{otevřený interval}, pokud existují otevřené
	intervaly $I_1, I_2, \ldots, I_n$ tak, že $I = I_1 \times \cdots \times I_n$.
\end{definice}

\begin{definice}
	Nechť $c \in \mathbb{R}^{n}$ a $r>0$. Definujeme \hldef{otevřenou kouli} jako 
	\[ B(c, r) = \left\{ x \in \mathbb{R}^{n} : |x - c|  
	= \sqrt{\sum_{i = 1}^{n} (x_i - c_i)^{2}} > r \right\} \]
\end{definice}

\begin{definice}
	Nechť $I \subset \mathbb{R}^{n}$ je otevřený interval a $f: I \to \mathbb{R}$ je funkce. Řekneme, že $f$ je
	\hldef{spojitá} v bodě $x_0 \in I$, pokud $\forall \epsilon > 0 \  \exists \delta > 0 \ 
	\forall x \in B(x_0, \delta) \cap I$ platí $|f(x) - f(x_0)| < \epsilon$.
	Řekneme, že $f$ je spojitá na $I$, pokud je spojitá ve všech bodech I.
\end{definice}

\begin{pozorovani}
	...
\end{pozorovani}

\begin{proof}
	Důkaz pozorování.
\end{proof}

\begin{dusledek}
	\(P(x,y)\) polynom dvou proměnných je spojitá funkce na \(\mathbb{R}^{2}\)
\end{dusledek}

\begin{vetat}
	\hlvetamath{Peano s $y^{(n)}$} (důkaz později, ne tento semestr)
	\\
	Nechť \(I \subset \mathbb{R}^{n+1}\) je otevřený interrval, \(f: I \to \mathbb{R}\) je spojitá, a \linebreak[1]
	\([x_0, y_0, \dots, y_{n-1}] \in I\). Pak \linebreak[1] \(\exists \delta > 0\) a v okolí \linebreak[1]
	\(x_0\) existuje interval \linebreak[1] \((x_0- \delta , x_0 + \delta)\) a funkce \(y(x)\) definována na 
	\linebreak[1] \((x_0 - \delta, x_0 + \delta)\) tak, že \(y(x)\) splňuje ODR \linebreak[1]
	\(y^{(n)}(x) = f(x, y(x^), \dots, y^{(n-1)}) \forall x \in (x_0 - \delta, x_0 + \delta)\) s počáteční podmínkou 
	\(y(x_0) = y_0, y'(x_0) = y_1, \dots , y^{(n-1)}(x_0) = y_{n-1}\).
\end{vetat}

\begin{pozn}
\begin{enumerate}
	\item tato věta je lokální a \(\delta\) může být velmi malé
	\item tato věta nedává jednoznačnost řešení
	\item každé řešení lze rozšířit do maximálního řešení
\end{enumerate}
\end{pozn}

\begin{definice}
	\(f: \mathbb{R} \to \mathbb{R}\) je \hldef{lipschtzovská}, 
	pokud \(\exists K > 0\), že \(|f(x) - f(y)| \leq K |x - y| \ \  \forall x,y \in \mathbb{R}\).
\end{definice}

\begin{definice}
	Nechť \(I \subset \mathbb{R}^{2}\) je otevřený interval. Řekneme, 
	že \(f: I \to \mathbb{R}\) je \hldef{lokálně lipschitzovský vůči y}, 
	pokud \(\forall U \subset I\) omezenou \(\exists K \in \mathbb{R}\), že
	\(|f(x,y) - f(x, \tilde{y})| \leq K |y - \tilde{y}| \ \ \forall [x, \tilde{y}] \in U\).
\end{definice}

\begin{vetat}
	\hlveta{Picard}
	\\
	Nechť \(I \subset \mathbb{R}^{2}\) je otevřený interval a \([x_0, y_0] \in I\). Nechť \(f: I \to \mathbb{R}\)
	je spojitá a lokálně lipschitzovská vůči y. Pak existuje \((x_0 - \delta, x_0 + \delta)\) a funkce
	\(y(x): (x_0 - \delta, x_0 + \delta) \to \mathbb{R}\) tak, že \(y(x)\) splňuje ODR
	\(y'(x) = f(x, y(x))\) pro \(x \in (x_0 - \delta, x_0 + \delta)\) s počáteční podmínkou \(y(x_0) = y_0\).
	\\
	Navíc y je jedinné řešení na \((x_0 - \delta, x_0 + \delta)\).
\end{vetat}